%!TEX root = ../dokumentation.tex

\pagestyle{empty}

\iflang{de}{%
% Dieser deutsche Teil wird nur angezeigt, wenn die Sprache auf Deutsch eingestellt ist.
\renewcommand{\abstractname}{\langabstract} % Text für Überschrift

% \begin{otherlanguage}{english} % auskommentieren, wenn Abstract auf Deutsch sein soll
\begin{abstract}
The goal of this student research project is the development of an audio player
for a Raspberry Pi, which should be realised in the programming language Go.
For development purposes a Raspberry Pi is provided. There are no restriction
on the usage of go libraries. The audio player enables the playback, pausing
and stopping of audio files as well as the management of a playlist and the
adjustment of the volume. The result of this work is an extensive audio player
for the Raspberry Pi with many functions, which can be used for many
operational purposes, e.g. for the adminstration of a theatre group.
\end{abstract}
}



%Das Ziel dieser Studienarbeit ist die Entwicklung eines Audioplayers für einen Raspberry Pi in der Programmiersprache Go. Der Audioplayer ermöglicht das Abspielen, Pausieren und Stoppen von Audiodateien sowie das führen einer Playlist und das Anpassen der Wiedergabelautstärke.
%Um dies zu realisieren wurde auf Basis eines bereitgestellten Raspberry Pi entwickelt und für die Umsetzung Bibliotheken für Go verwendet, die z.B. das dekodieren einer MP3 Datei ermöglichen.
%Das Ergebnis dieser Arbeit ist ein umfangreicher Audioplayer für den Raspberry Pi mit vielen Funktionen, welcher für viele Einsatzzwecke wie z.B. für die Theatergruppe genutzt werden kann.
