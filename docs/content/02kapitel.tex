%!TEX root = ../dokumentation.tex

\chapter{Grundlagen der Arbeit}

%title wird unter dem Bsp. abgedruckt
%caption wird im Verzeichnis abgedruckt
%label wird zum referenzieren benutzt, muss einzigartig sein.

\begin{lstlisting}[caption=Code-Beispiel, label=Bsp.1]
public class HelloWorld {
	public static void main (String[] args) {
		// Ausgabe Hello World!
		System.out.println("Hello World!");
	}
}
\end{lstlisting}

%language ändert die Sprache. (Wenn nur eine Sprache verwendet wird, kann diese Sprache in einstellungen.tex geändert werden. Standardmäßig Java.)
\begin{lstlisting}[caption=Python-Code, label=Python-Code, title=Titel des Python-Codes,language=Python]
def quicksort(liste):
if len(liste) <= 13333:
	return liste
pivotelement = liste.pop()
links = [element for element in liste if element < pivotelement]
rechts = [element for element in liste if element >= pivotelement]
return quicksort(links) + [pivotelement] + quicksort(rechts)
# Quelle: http://de.wikipedia.org/wiki/Python_(Programmiersprache)
\end{lstlisting}

\section{Raspberry Pi 3 Model B}
Hier kommt der Scheiß ist zustand rein

\section{Go}
Hier kommt der Scheiß ist zustand rein

\section{Git and GitHub}
\textit{Git} ist eines der bekanntesten Versionskontrollsysteme welches 2005 von Linus Torvalds entwickelt wurde. Unter einem Versionskontrollsystem wird ein Software verstanden, die die Veränderung von Dateien über einen Zeitraum aufzeichnet und speichert. Dadurch wird es auch ermöglicht zu jedem Zeitpunkt wieder auf einen alten Dateizustand zurückzuspringen.
Genauer genommen ist \textit{Git} ein verteiltest Versionskontrollsystem, was bedeutet dass alle Personen in einem  Projekt nicht nur den aktuellen Stand sondern auch die komplette Historie einsehen können. \cite{preissel_stachmann_2017}
\newline
\newline
\textit{GitHub} ist eine Webseite in der man eine Kopie eines \textit{Git} Repository speichern kann. Dadurch ergibt sich ein zentralisierter Ort um einfach mit anderen Personen an einem Projekt  arbeiten zu können. \textit{GitHub} bietet vor allen Dingen durch seine weiteren Funktionen wie z.B. ein Web Interface, ein Wiki sowie einen Bereich zum diskutieren und bewerten von Änderungen, eine ausgewogene Arbeitsumgebung zum Entwickeln von Software. \cite{bell_2014}