%!TEX root = ../dokumentation.tex

\chapter{Aufgabenstellung}

Im Rahmen der Studienarbeit soll ein Musikplayer für den Raspberry Pi entworfen
werden. Die Realisierung soll mit der Sprache Go erfolgen. Als Grundlage können
alle bereits bestehenden Musikplayer dienen. Auch gibt es keine
Einschränkungen, was den Einsatz von Bibliotheken zur Musiksteuerung oder
ähnliches angeht. \hfill \break

Der entstehende Musikplayer soll später von einem bestehenden Tool aus benutzt
werden. Deshalb ist es wichtig, dass der Player eine Schnittstelle bietet, um
das Aufrufen von externen Programmen ohne weiteres zu ermöglichen. \hfill
\break

Funktionstechnisch sollen folgende Feature auf jeden Fall implementiert werden:
\begin{itemize}
\item Starten,Stoppen und Pausieren der Wiedergabe der Audiodatei
\item Verändern der Ausgabelautstärke
\item Neustarten und überspringen von Audiodateien
\item Audiodateien in eine Playlist hinzufügen und Verwalten
\item Fade-In und Fade-Out Funktionalität ermöglichen
\end{itemize}

