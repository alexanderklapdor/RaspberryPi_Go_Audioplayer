%!TEX root = ../dokumentation.tex

\chapter{Aufgabenstellung}
Die Aufgabe im Rahmen der Studienarbeit umfasst die Entwicklung eines
Audioplayers für den Raspberry Pi. Die Realisierung dieses Players soll in der
Sprache Go erfolgen. Als Orientierungshilfe können alle bereits bestehenden Audioplayer
für den Raspberry Pi dienen. Auch gibt es keinerlei Einschränkungen was den
Einsatz von Bibliotheken oder ähnlichem bei der Entwicklung angeht. \hfill
\break

Wichtig ist, dass der entstehende Audioplayer zukünftig von einem externen Tool
aus angesteuert werden kann. Deshalb muss der Player eine Schnittstelle bieten,
welche einen Informationsaustausch unterstützt. Auch ist gefordert, dass
Audiodateien im Hintergrund abgespielt werden können, um parallel ein
störungsfreies Arbeiten im Vordergrund zu ermöglichen. \hfill \break
Der Fokus soll vor allem auf Funktionen wie dem Starten und Stoppen von
Audiodateien gesetzt werden. Erstrebenswert wäre weiterhin die Umsetzung
möglichst vieler der folgenden Funktionen: 

\begin{itemize}
\item Starten, Stoppen und Pausieren von Audiodateien
\item Verändern der Ausgabelautstärke
\item Neustarten und Überspringen von Audiodateien
\item Hinzufügen und Verwalten von Audiodateien in einer Playlist 
\item Bereitstellen einer Fade-In und Fade-Out Funktionalität 
\end{itemize}

