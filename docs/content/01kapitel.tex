%!TEX root = ../dokumentation.tex

\chapter{Einleitung}
Aktuell kann der Trend beobachtet werden, dass etablierte Applikationen oder
Programme dazu übergehen, zusätzliche Funktionalitäten zu ihren Grundfeaturen
hinzuzufügen. So umfasst die beliebteste Handyapplikation in China namens
\textit{WeChat} weitaus mehr Möglichkeiten als nur zum einfachen
Nachrichtenaustausch, wofür sie anfänglich entwickelt wurde. Mittlerweile
bietet die Applikation sogar Möglichkeiten zum Anschauen von Landkarten, zum
Einkaufen von Artikeln, zum Managen von Flügen oder auch zur Verwaltung von
Zahlungen. Auch WhatsApp hat begonnen ihren Fokus auf zusätzliche Funtionen zu
erweitern. So stellt WhatsApp mittlerweile Möglichkeiten für Internetanrufe und
Videochat zur Verfügung, und arbeitet aktuell an der Einführung eines
Bezahl-Features. \break

Diese Applikationen erübrigen zwar die Installation von vielen zusätzlichen
Programmen, sind aber nicht immer für alle Anwendergruppen so attraktiv. Als
Nachteil an diesem Vorgehen ist zu nennen, dass diese Applikationen und
Programme sich immer mehr zu \glqq Allrounderwerkzeugen \grqq entwickeln, und
dadyrcg speichertechnisch immer größer werden. Des Weiteren bilden die
zusätzlichen Funktionalitäten oftmals nur Standardfunktionen ab, und ersetzen
so keinesfalls andere Progamme, welche sich nur auf ein bestimmtes Feature
spezialisiert haben. \break

Hier schon mal Aufgabenrichtung anreißen.
\begin{itemize}
\item Applikation zum Abspielen von Musik
\item spezialisiertes Programm auf Kernfeature des Musik abspielens
\end{itemize}


\section{Ist-Zustand}
Kp hier brauche ich nochmal die genaue Beschreibung von dem Dozenten. \newline
- Musik wird derzeit wie abgespielt? Nicht über Rapi oder?

\section{Soll-Zustand}
\begin{itemize}
\item Theatergrupppe
\item RasberryPi
\item Musik abspielen
\item Lichtsteuerung bereits über RaspberryPi
\item Programmablauf integrieren
\item Automatisierung
\end{itemize}

\section{Aufgabe}
Im Rahmen dieser Studienarbeit soll ein Programm entwickelt werden, welches
die grundsätzlichen Funktionalitäten eines Musikplayers auf einem RaspberryPi
ermöglicht. Dieses Programm soll einfach gehalten werden und eine Schnittstelle
bieten, um von einem einem bestehenden externen Programm aus aufgerufen werden
kann. 
